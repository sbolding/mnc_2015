Because we have only considered problems with constant densities and heat capacities, the following
derivation is depicted in terms of temperature $T$ rather than material energy
for simplicity.  Application of the first order Taylor expansion in time of the
gray emission source $B(T)$, about some temperature $T^*$ at some
time near $t^{n+1}$ gives
\begin{equation}\label{new_planck}
    \sigma_a^* a c T^{4,n+1} \simeq \sigma_a^* a c \left[T^{*4} + (T^{n+1} - T^*) 4T^{*3} \right]
\end{equation}
where $\sigma_a^*$ is evaluated at $T^*$.  Substitution of this into the material
energy equation given by Eq.~\eqref{mat_eq} yields
\begin{equation}
    \rho c_v \left( \frac{T^{n+1} - T^{n}}{\Delta t} \right) = \sigma_a^* \phi^{n+1} -
    \sigma_a^* a c \left[ T^{*4} +  (T^{n+1} - T^*) 4T^{*3} \right].
\end{equation}
Further manipulation to solve for the unkown temperature at the next time step $T^{n+1}$
yields
\begin{align*}
    \rho c_v \left( \frac{T^{n+1} - T^* + T^* - T^n}{\Delta t} \right) 
    &= \sigma_a^* \phi^{n+1} - \sigma_a^* a c \left[ T^{*4} +  (T^{n+1} - T^*) 4T^{*3}
\right] \\
\left( {T^{n+1} - T^*} \right) \left\{\rho c_v + \sigma_a^* a c 4
T^{*3} \Delta t \right\} &= \Delta t \sigma_a^* \phi^{n+1} - \Delta t \sigma_a^* a c
T^{*4} + \rho c_v (T^n - T^*) \\
\left( T^{n+1} - T^* \right) &= \frac{\left( \Delta t \sigma_a^* \phi^{n+1} - \Delta t \sigma_a^* a c
T^{*4} + \rho c_v (T^n - T^*) \right)}{\left\{\rho c_v + \sigma_a^* a c 4
T^{*3} \Delta t \right\}} \\
\left( T^{n+1} - T^* \right) &= \frac{ {\displaystyle \frac{\sigma_a^* \Delta t}{\rho
c_v}}  \left[ \phi^{n+1} -  a c T^{*4} \right] + (T^n - T^*) }{1 +
        \sigma_a^* a c \Delta t\frac{\displaystyle 4
T^{*3}}{\displaystyle \rho c_v } }.
\end{align*}
This provides an expression for $T^{n+1}$ as a
function of $T^*$ and the radiation scalar intensity $\phi^{n+1}$, i.e.,
\begin{equation}
\label{lo_t_new}
T^{n+1}  = \frac{1}{\rho c_v } f\sigma_a^* \Delta t \left( \phi^{n+1} - c a T^{*4} \right)
+ f T^n + (1-f) T^*.
\end{equation}
If $T^*=T^n$, then this produces the standard IMC temperature update equation.  The
expression for $T^{n+1}$ can be substituted back into Eq.~\eqref{new_planck} to form
an explicit approximation for the emission source at $t_{n+1}$ as
\begin{equation}\label{t_next1}
    \sigma_a a c T^{4,n+1} \simeq \sigma_a^* (1 -f^*) \phi^{n+1}
    + f^* \sigma_a^* a c T^{4,n} + \rho c_v\frac{1-f^*}{\Delta t} (T^N - T^*)
\end{equation}
where $f^* = (1 + \sigma_a^* c \Delta t \beta^*)^{-1}$ is the usual Fleck factor with
\begin{equation}
    \beta^* = \frac{4 a T^{*3}}{\rho c_v}
\end{equation}
The material temperature is updated at the end of the time step using Eq.~\ref{lo_t_new}.

Now, the above equation for $U_r^{n+1}$ must be discretized in space.  Taking the
left spatial moment yields
\begin{align}\label{temp_moms}
    \mom{\sigma_a^* a c T^{4,n+1}}_L =& \frac{2}{h_i} \int_{x_\il} ^{x_{\ir}}
    b_L(x) \left[ \sigma_a^*(1-f^*)\phi^{n+1} + f^* \sigma_a^* a c T^{4,n}\right] \d x.
\end{align}
To keep the derivation general, we look at the two terms on the right side
seperately: $f^* \sigma_a^* a c T^{4,n}$ can be evaluated explicitly because the spatial
dependence over a cell of $T^{4,n}$ is already assumed LD. 
 The $\phi^{n+1}$ term is divided and multiplied by a normalization integral
to form the appropriate average
\begin{equation}
    \left\{ \displaystyle \frac{ \displaystyle \frac{2}{h_i} \int_{x_\il} ^{x_{\ir}}
    b_L(x) \sigma_a^*(1-f^*)\phi^{n+1}}{\displaystyle\frac{2}{h_i} \int_{x_\il} ^{x_{\ir}} b_L(x)
\phi(x) \d x} \right\} \int_{x_\il} ^{x_{\ir}} b_L(x)  \phi(x) \d x = \mom{\sigma_a^*
(1-f^*) }_L \mom{\phi^{n+1}}_L
\end{equation}
where the quotient in braces is defined as $\mom{\sigma_a^*(1-f^*)}_L$ and
$\mom{\phi}^{n+1}_L= \mom{\phi}_L^{+,n+1}+\mom{\phi}_L^{-,n+1}$ is in terms of
the desired unknowns. 

The evaluation of $\mom{\sigma_a^*(1-f^*)}_L$ is performed
based on the assumed spatial dependence over a cell of $\sigma_a$ and $f^*$.  If $f^*$
and $\sigma_a^*$ are assumed constant over a cell (i.e., the dependence of these terms
on temperature is assumed constant over a cell), Eq.~\eqref{temp_moms} reduces to
\begin{equation}\label{temp_const}
    \mom{\sigma_a^* a c T^{4,n+1}}_L = \sigma^*_{a}(1-f^*)\mom{\phi^{n+1}}_L +
    f^*
    \sigma^*_{a} \mom{ a c T^{4,n}}_L
\end{equation}
where 
\begin{equation}
    f^* = f(T^*_{avg,i}) = f\left(\sqrt[4]{\frac{T^{*4}_{L,i}+T^{*4}_{R,i}}{2}}\right)
\end{equation}
and $\sigma^* = \sigma_a(T^*_{avg,i})$.  Constant cross sections over a cell
have been assumed for the first iteration of the code.   If an LD
dependence of the cross sections and temperature is used, $\mom{\sigma_a^*(1-f^*)}_L$
can be evaluated without truncation via a Gaussian quadrature or analytical
integration and use of the previous (in the HOLO context) HO fine mesh solution; this
is the same procedure used for evaluation of the consistency terms.  It will likely
be easier to use the full non-linear NK solve to handle the LD case than to try to
define an LD dependence of $f$.

Based on a guess for $T^*$, the above equation gives an expression for
the Planckian emission source on the right hand side
of Eq.~\eqref{lo_tran} with an additional effective scattering source.
This allows for four linear equations for the four remaining radiation unknowns to be fully
defined.  The final equation for the left basis moment and positive flow, for constant
$f^*$ and $\sigma_a^*$ over a cell, becomes
\begin{multline}\label{lo_trans_fin}
    -2{\mu}_{i-1/2}^{n+1,+} \phi_{i-1/2}^{n+1,+} + \mom {\mu}_{L,i}^{n+1,+}
  \mom{\phi}_{L,i}^{n+1,+}
  +  \mom\mu_{R,i}^{n+1,+}
  \mom{\phi}_{R,i}^{n+1,+} +  \left(\sigma_t^*+\frac{1}{c \Delta t} \right) h_i 
  \mom{\phi}_{L,i}^{n+1,+} \\- \frac{h_i}{2}\left(\sigma_s^* + \sigma_a^*(1-f^*)\right)
  \left( \mom{\phi}_{L,i}^{n+1,+} +
  \mom\phi_{L,i}^{n+1,-}\right) = \\ \frac{1}{2} h_i \sigma_a^*a c f^* \left(\frac{2}{3} T_{L,i}^{4,n} +
  \frac{1}{3} T_{R,i}^{4,n} \right) + 
  \frac{h_i}{c\Delta t}\mom{\phi}_{L,i}^{n,+}
\end{multline}
%In operator
%notation, denote the LO system in operator notation to be
%\begin{equation}
%    \B D(T) \Phi = \B B(T),
%\end{equation}
%where $\B B$ denotes the emission source and $\B D$ accounts for the effective
%scattering source, as a function of temperature.

THIS SECTION IS NOT COMPLETELY RIGHT

Once these linear equations have been solved for $\phi^{n+1}$, a new estimate of
$T^{n+1}$ can be determined.  To conserve energy, the same linearization used to
solve the radiation equation must be used in the material energy equation.
Substitution of
Eq.~\eqref{temp_const} into the material energy equation yields
\begin{equation} 
    \rho c_v\frac{ T^{n+1}-T^n}{\Delta t} = \sigma_a^* \phi^{n+1} - \left(\sigma_a^*  (1 -f) \phi^{n+1}
    + f \sigma_a^* a c T^{4,n} \right),
\end{equation}
which gives a temperature at the end of the time step as
\begin{equation}\label{new_temp}
    T^{n+1}= \frac{f^* \sigma_a^* \Delta t}{\rho c_{v}}  \left( \phi^{n+1}
    - c a T^{n4}\right)
    +  T^n,   
\end{equation}
This is how the IMC method estimates the temperature at the end of the time
step, following the MC solve.  Here, the LO radiation equations have taken the place of the Monte Carlo
solve.  To account for spatial dependence, the above equation can simply be evaluated
using $\phi^{n+1}_L$ and $T^*_L$ to get $T_L^{n+1}$.

Based on these equations, the algorithm for solving the LO system with constant $f^*$
and cross sections over a cell is defined as
\begin{enumerate}
    \item Guess $T^*_L$ and $T^*_R$, typically using $T^n$.
    \item  Build the LO system based on the effective scattering $(1-f^*)$ and emission terms
        (i.e., evaluation of  Eq.~\eqref{lo_trans_fin}).
    \item Solve the linearized LO system to produce an estimate for $\phi^{n+1}$.
    \item Evaluate a new estimate of the $T_{L,i}$ and $T_{R,i}$ at the end of the time step
    $\tilde{T}^{n+1}$ using Eq.~\eqref{new_temp}.
    \item $T^*\leftarrow\tilde{T}^{n+1}$.
    \item Repeat 2-5 until $\tilde T^{n+1}$ and $\phi^{n+1}$ are converged.
\end{enumerate}
Because of the chosen linearization, the convergence primarily takes place in the
lagged material properties and factor $f$.
